\justify
\skipspace
The environmental impact of global warming has accelerated the interest to adopt non-conventional power generation \cite{EPFL}. There are several promising sustainable energy alternatives, among which the adoption of Thermoelectric (TE) materials to scavenge the by-product heat generation is widely accepted \cite{EPFL}. It is crucial for applications associated with energy harvesting to possess a high figure of merit ZT ($\geq$ 1) \cite{EPFL}.
\justify
\skipspace
The ZT can be related to other thermoelectric parameters by ZT =($\frac{\sigma_e \times S_B^2 \times T}{\kappa_{ph} + \kappa_e}$), where $\sigma$, $S_B$, $\kappa_{ph} + \kappa_{e}$, and T are the electrical conductivity, Seebeck coefficient, total thermal conductivity, and temperature value respectively \cite{EPFL}.
\justify
\skipspace
Experimental and theoretical identification of two-dimensional (2D) \cite{EPFL} or three-dimensional (3D) efficient TE materials is laborious and time inefficient \cite{EPFL}. It is also a colossal task to compile databases of thermoelectric parameters for various synthesized TE materials and their variations with doping (n-type or p-type) \cite{EPFL}. Computational methods using density functional theory (DFT) are also time-consuming and demand high computational complexity for exploring TE materials \cite{EPFL}.
\newpage
\justify
\skipspace
Efficient TE materials require a large ZT which in turn requires to maximize the Seebeck coefficient absolute value, minimize the thermal conductivity and possess a high electrical conductivity. Optimizing these parameters is a complicated task as they are inherently dependent and conflicting in nature \cite{EPFL}. Thus, optimizing AT requires a thorough understanding of these various transport properties and their interrelated material characteristics.
\justify
\skipspace
The Seebeck coefficient depends on this energy-dependent conductivity\\
around a fermi window centered about the fermi energy level,
which is given by the Mott expression (Eq. \ref{Seebeck}) \cite{EPFL}.

\begin{center}
    \begin{equation}
        S_B = \frac{\pi^2}{3}(\frac{{K_B}^2\times T}{q})[\frac{d[ln(\sigma (E))]}{dE}]_{E=E_F} = (\frac{8\pi^2{K_B}^2T}{3qh^2})m_d^*(\frac{\pi}{3n})^{2/3}
        \label{Seebeck}
    \end{equation}
\end{center}
\justify
\skipspace
where n is the carrier concentration and effective mass $m_d^*$ of the carrier when present in the conduction band or valence band. This effective mass ($m_d^*$) is obtained from the function of the density of states (DOS) and is thus also known as $m_{DOS}^*$ \cite{EPFL}. The underlying assumption for the final closed form expression is the presence of a parabolic band and an energy-independent scattering approximation \cite{EPFL}. The electrical conductivity ($\sigma_e$) can be approximated by the Drude model in terms of its carrier concentration (n) and mobility ($\mu$) as shown in EQ. \ref{Eqn: 1.3}. Thus, the influence of carrier concentration impacts both the parameters contradictory as shown in Fig. \ref{Fig: 1.1}.

\justify
\skipspace
\begin{figure}[!ht]
    \centering
    \includegraphics[scale=0.25]{images/logo.png}
    \caption{Figure 1.1}
    \label{Fig: 1.1}
\end{figure}

\begin{center}
    \begin{equation}
        \sigma_e  = nq\mu = \frac{nq^2\tau}{m}
        \label{Eqn: 1.2}
    \end{equation}
\end{center}

\begin{center}
    \begin{equation}
        \kappa  = \kappa_{ph} + \kappa_e = (\frac{\pi^2}{3})(\frac{nK_B^2T\tau}{m}) + L_n\times \sigma_eT
        \label{Eqn: 1.3}
    \end{equation}
    \begin{equation}
        L_n \approx (\frac{\pi^2}{3})(\frac{K_B}{q})^2
        \label{Eqn: 1.4}
    \end{equation}
\end{center}

\section{Section 1.1}
\subsection{Sub-Section 1.1.1}
\justify
\skipspace Content.

\subsection{Sub-Section 1.1.1}
\justify
\skipspace Content.

\newpage


