\section{CPU Scheduling: FCFS}
\label{sec:cpu-scheduling}

Implementation of CPU scheduling in a first come first serve (FCFS) manner.

\inputminted[linenos,autogobble]{c}{code/fcfs.c}

The scheduler selects the process that has been waiting the longest.

\begin{lstlisting}[style=output]
Processes  Burst Time   Waiting Time    Turn Around Time
1	   10		0		10
2	   5		10		15
3	   8		15		23
Average waiting time = 8.333333
Average turn around time = 16.000000
\end{lstlisting}

\section{CPU Scheduling: Priority}
\label{sec:cpu-scheduling-priority}

Implementation of CPU scheduling using a ``priority'' based approach using assigned ranks/priority algorithms.

A major problem with priority scheduling is indefinite blocking or starvation. A solution to the problem of indefinite blockage of the low-priority process is aging. Aging is a technique of gradually increasing the priority of processes that wait in the system for a long period of time.

\inputminted[linenos,autogobble]{c}{code/priority.c}

\begin{lstlisting}[style=output]
Order in which processes gets executed
2 3 4 1 
Process         BurstTime      WaitingTime     TurnAroundTime
2		19		0		19
3		27		19		46
4		25		46		71
1		100		71		171
Average waiting time = 34.000000
Average turn around time = 76.750000
\end{lstlisting}